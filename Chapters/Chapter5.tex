\chapter{Kết Luận- Hướng Phát Triển Đề Tài}
\label{Chapter0} % For referencing the chapter elsewhere, use \ref{Chapter1} 
\rhead{\bfseries Khóa Luận Tốt Nghiệp,Năm 2013}
\section{Kết Luận}
Trên đây tôi đã trình bày phương pháp tính toán cấu trúc vùng năng lượng của chất bán dẫn bằng phương pháp kp $14\times14$ chính xác và trong gần đúng nhiễu loạn L$\ddot{o}$wdin.
Sau khi có được năng lượng thì tôi đã ứng dụng nó vào phương trình Bloch bán dẫn để tìm dòng điện tích, dòng spin cũng như độ phân cực do giới hạn về thời gian làm khóa luận cũng như công cụ tính toán nên kết quả về dòng spin, dòng điện tích chưa được mô phỏng hóa qua dữ liệu.
\section{Hướng phát triển đề tài}
Tôi sẽ tiếp tục phát triển giải bài toán trong trường hợp có từ trường cũng như điện trường vào, và sẽ giải số phương trình (5.20) để tính các dòng, trong đó có các dòng đã có trong thực nghiệm đo khi xét từ trường ngoài vào ở hai bài báo sau: Magnetoelectric Photocurrent Generated by Direct Interband Transitions in InGaAs/InAlAs Two-Dimensional Electron Gas(PR.L 104, 246601 (2010)) và
Quadratic magnetic field dependence of magnetoelectric photocurrent (PR.B 83, 155307 (2011))

