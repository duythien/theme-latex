\chapter{Lời Mở đầu}
\label{Chapter0} % For referencing the chapter elsewhere, use \ref{Chapter1} 
\rhead{\bfseries Khóa Luận Tốt Nghiệp,Năm 2013}Ngày nay, hầu hết các công nghệ sử dụng trong các lĩnh vực công nghệ thông tin và điện tử là kết quả của quá trình ứng dụng tương tác của trường điện từ với vật chất, một dạng tương tác giữa vật chất và ánh sáng. Chúng ta có thể thấy chúng hiện diện khắp nơi quanh cuộc sống của chúng ta chẳng hạn như các đèn LED dùng trong hiển  thị thông tin trong quảng cáo hay thiết bị gia dụng điện, hay xa hơn là các cổng giao tiếp quang học trong mạng lưới cáp quang truyền dẫn internet tốc độ cao, cáp dẫn điện và lĩnh vực đo đạc cần độ chính xác cao$\dots$Ngoài ra trong thời gian tới sẽ ứng dụng rất mạnh vào công nghệ  lưu trữ và xử lý thông tin đặc biệt là công nghệ spin điện tử (spintronics). Do đó để hiểu được các tính chất quan học của chất bán dẫn đều quan trong đầu tiên là ta phải tính được cấu trúc vùng năng lượng.\\
Trong luận văn này, tôi sẽ trình bày phương pháp tính cấu trúc vùng năng lượng của chất bán dẫn bằng phương pháp kp. Sau khi có được cấu trúc vùng năng lượng rồi ta sẽ ứng dụng nó vào phương trình Bloch bán dẫn để tìm dòng điện tích, dòng spin cũng như độ phân cực của điện tử và lỗ trống.
